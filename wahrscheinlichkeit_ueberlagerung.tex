\section{Wahrscheinlichkeiten bei Überlagerung}

\textbf{AWGN-Kanal}\\
Additives weißes Rauschen, mittelwertfrei


\textbf{Nutzsignal (useful signal):}
\begin{align}
p_U(u)
\end{align}

\textbf{Rauschsignal (noise signal):}
\begin{align}
p_N(n)
\end{align}

\textbf{Empfangssignal (received signal):}
\begin{align}
p_Y(y)
\end{align}

\textbf{Überlagerung wird zur Faltung:}\\
Addition von Nutzsignal und Rauschen:
\begin{align}
y = u + n \Rightarrow p_Y(y) = \int p_U(u) \cdot p_N(y-u) \, du = (p_U * p_N)(y)
\end{align}

Die Addition zweier Zufallsvariablen wird zur Faltung ihrer Wahrscheinlichkeitsdichtefunktionen.

\textbf{Fehlerwahrscheinlichkeit:}
\begin{align}
P_{\text{Fehler}} = \int p_E(e)\,de
\end{align}
