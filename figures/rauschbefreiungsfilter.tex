\begin{figure}[!htb]
\centering
\begin{tikzpicture}[
    auto, 
    node distance=2cm,
    block/.style={rectangle, draw, minimum height=0.8cm, minimum width=1.2cm},
    mult/.style={rectangle, draw, minimum height=0.8cm, minimum width=0.8cm},
    sum/.style={circle, draw, inner sep=0pt, minimum size=0.7cm},
    >=latex
]
% Neues Eingangssignal X(n) und Noise Addition
\node at (-2,0.3) {$X(n)$};
\node[sum] at (-0.5,0) (noise_sum) {$+$};
\draw[->] (-2,0) -- (noise_sum.west);

% Rauschsignal von unten
\node at (-0.5,-1.3) {$N(n)$};
\draw[->] (-0.5,-1) -- (noise_sum.south);

% Hauptleitung
% Eingangssignal und Verzögerungselemente
\node at (1,0.3) {$Y(n)$};
% Positionierung der Verzögerungselemente
\node[block] at (2.5,0) (delay1) {$z^{-1}$};
\node[block] at (5,0) (delay2) {$z^{-1}$};
\node at (7,0) (dots) {$\cdots$};
\node[block] at (9,0) (delayN) {$z^{-1}$};
% Durchgehende Pfeile, die direkt zu den Boxen führen
\draw[->] (noise_sum.east) -- (delay1.west);
\draw[->] (delay1.east) -- (delay2.west);
\draw[->] (delay2.east) -- (6.4,0);
\draw[->] (7.6,0) -- (delayN.west);
% Multiplizierer mit X Symbol
\node[mult] at (1,-2) (weight0) {$\times$};
\node at (1.8,-2) {$h_0$};
\node[mult] at (3.5,-2) (weight1) {$\times$};
\node at (4.3,-2) {$h_1$};
\node[mult] at (6,-2) (weight2) {$\times$};
\node at (6.8,-2) {$h_2$};
\node[mult] at (10,-2) (weightN) {$\times$};
\node at (10.8,-2) {$h_N$};
% Summierer
\node[sum] at (1,-4) (sum0) {$+$};
\node[sum] at (3.5,-4) (sum1) {$+$};
\node[sum] at (6,-4) (sum2) {$+$};
\node at (7.5,-4) (dots2) {$\cdots$};
\node[sum] at (10,-4) (sumN) {$+$};
% Ausgang
\node at (11.5,-4) (output) {$Z(n)$};
% Vertikale Verbindungen
\draw[->] (1,0) -- (weight0);
\draw[->] (3.5,0) -- (weight1);
\draw[->] (6,0) -- (weight2);
\draw[->] (delayN.east) -- ++(0.4,0) -- ++(0,-1.2) -- (10,-1.2) -- (weightN.north);
% Verbindungen zu Summierern
\draw[->] (weight0) -- (sum0);
\draw[->] (weight1) -- (sum1);
\draw[->] (weight2) -- (sum2);
\draw[->] (weightN) -- (sumN);
% Verbindungen zwischen Summierern
\draw[->] (sum0) -- (sum1);
\draw[->] (sum1) -- (sum2);
\draw[->] (sum2) -- (dots2);
\draw[->] (dots2) -- (sumN);
\draw[->] (sumN) -- (output);
\end{tikzpicture}
\caption{Rauschbefreiungsfilter}
\end{figure}

\FloatBarrier
