\section{Korrelation}

\textbf{Zusammenfassung der Korrelationseigenschaften:}

\begin{itemize}
  \item \textbf{Unkorreliert (Nicht korreliert):} 
    Prozesse haben keine lineare Abhängigkeit.
    \begin{align}
      C_{XY} &= 0 \text{ oder } R_{XY}(0) = \mathbb{E}[XY] = \mathbb{E}[X] \cdot \mathbb{E}[Y]
    \end{align}

    
  \item \textbf{Orthogonal:}
    Prozesse haben kein gemeinsames Leistungsspektrum.
    \begin{align}
      R_{XY}(0) &= 0
    \end{align}
    
  \item \textbf{Statistisch unabhängig:} 
    Stärkste Form der Unabhängigkeit. Stochastische Prozesse beeinflussen sich gegenseitig in keiner Weise.
    \begin{align}
      p_{XY}(x,y) = p_X(x) \cdot p_Y(y)
    \end{align}
   
    
  \item \textbf{Wichtige Zusammenhänge:}
    \begin{align}
      &\text{Orthogonal} \Leftrightarrow  \text{Unkorreliert + mittelwertfrei}\\
      &\text{Orthogonal (z.B.} Y=X^2 \text{für mittelwertfreies, symmetrisches X)} \not\Rightarrow \text{Statistisch unabhängig}\\
      &\text{Statistisch unabhängig} \Rightarrow \text{Nicht korreliert}\\
      &\text{Aus statistischer Unabhängigkeit folgt keine Aussage über Orthogonalität}
    \end{align}
    \textbf{Merksatz:} Die Korrelation erfasst nur lineare Zusammenhänge zwischen stochastischen Prozessen. Für nichtlineare Abhängigkeiten ist sie kein geeignetes Maß.
\end{itemize}



\textbf{Korrelation:}
\begin{align}
C_{XY} &= \text{Cov}(X,Y) = \mathbb{E}[(X-\mu_X)(Y-\mu_Y)] = \mathbb{E}[XY] - \mu_X \mu_Y
\end{align}

\textbf{Korrelationskoeffizient:}
\begin{align}
\rho_{XY} &= \frac{C_{XY}}{\sigma_X \cdot \sigma_Y}
\end{align}

\textbf{Kreuzkorrelation:}
\begin{align}
R_{XY}(k) &= \mathbb{E}[X(n)Y(n+k)]
\end{align}

\begin{align}
R_{XY}(0) &= \mathbb{E}[XY]
\end{align}

\begin{align}
R_{XY}(k) &= R_{YX}(-k)
\end{align}

\textbf{Autokorrelationsfunktion:}
\begin{align}
R_{XX}(k) &= \mathcal{F}^{-1}[S_{XX}(\Omega)] = \frac{1}{2\pi} \int_{-\pi}^{\pi} S_{XX}(\Omega) e^{j k \Omega} d\Omega
\end{align}
\begin{align}
R_{XX}(0) =E[X^2(n)] = P_X(\Omega) \neq P_X
\end{align}

\text{normierte AKF:}
\begin{align}
\rho_{XX}(k) = \frac{R_{XX}(k)}{R_{XX}(0)}
\end{align}

\textbf{Leistungsdichtespektrum:}
\begin{align}
R_{XX}(k) \overset{\mathcal{F}}{\circ\mbox{---}\bullet} S_{XX}(\Omega)
\end{align}

\begin{align}
S_{XX}(\Omega) = \mathcal{F}\{R_{XX}(k)\}; R_{XX}(k) &=  \mathcal{F}^{-1}\{S_{XX}(\Omega)\}
\end{align}

\begin{align}
R_{XX}(0) = \mathcal{F}^{-1}\{S_{XX}(\Omega)\}|_{\tau=0} = \frac{1}{2\pi} \int_{-\pi}^{\pi} S_{XX}(\Omega) \, d\Omega
\end{align}

Gibt an, wie die Leistung eines stochastischen Signals über verschiedene Frequenzen verteilt ist. Die Gesamtleistung durch Integration berechenbar.

\begin{align}
\mathcal{F}\{\delta(k)\} &= 1 \\
\mathcal{F}\{\delta(k+1)\} &= e^{j\Omega} \\
\mathcal{F}\{\delta(k-1)\} &= e^{-j\Omega}
\end{align}

Kreisfrequenz im kontinuierlichen Fall: $\Omega = 2 \pi f$

\begin{align}
e^{j k \Omega} = \cos(k \Omega) + j \cdot \sin(k \Omega)
\end{align}

Für symmetrisches Intervall
\begin{align}
\int_{-a}^{a} e^{j k \Omega} d\Omega = \int_{-a}^{a} \cos(k \Omega) d\Omega
\end{align}

\textbf{Signal Noise Ratio (SNR):}
\begin{align}
\text{SNR} &= \frac{P_X}{P_N} = \frac{R_{XX}(0)}{R_{NN}(0)} = \frac{\int_{-\pi}^{\pi} S_{XX}(\Omega) \, d\Omega}{\int_{-\pi}^{\pi} S_{NN}(\Omega) \, d\Omega}\\
\end{align}
